\section{Materials and methods}

%%%%%%%%%%%%%%%%%%%%%%%%%%%%%%%%%%%%%%%%%%%%%%%%%%%%%%%%%
\subsection{Numerical algorithm}

%\subsubsection{Equations of motion and discretization} 
The isothermal filtered Navier-Stokes equations are solved in their rotational form \citep{Orszag1975}, to ensure conservation of mass and kinetic energy: 
%
\begin{equation}
\begin{cases}
   \frac{\partial \tilde{u}_i}{\partial t} 
      + \tilde{u}_j ( \frac{\partial \tilde{u}_i}{\partial x_j} 
      - \frac{\partial \tilde{u}_j}{\partial x_i} ) 
      = 
      - \frac{\partial \tilde{\pi}}{\partial x_i} 
      - \frac{\partial \tau_{ij}^{SGS}}{\partial x_j} 
      - \Pi_1  + \tilde{f}_i^{\Gamma_{\mathrm{b}}} & \text{in $\Omega \times [0,T]$} \, , \\
   	\frac{\partial \tilde{u}_i}{\partial x_i} =0 & \text{in $\Omega \times [0,T]$} \, , \\
	\frac{\partial \tilde{u}}{\partial z} = \frac{\partial \tilde{v}}{\partial z} = 
	\tilde{w} = 0 & \text{in $\Gamma_{\mathrm{top}} \times [0,T]$} \, , \\
	(\mathbf{\tilde{u}} \cdot \mathbf{\tilde{n}}) \mathbf{\tilde{n}} = \mathbf{\tilde{u}}_N = 0 & \text{in $\Gamma_{\mathrm{b}} \times [0,T]$} \, , \\
	\mathbf{\tilde{t}} = - \left( \frac{ \kappa ( \mathbf{\tilde{u}} - \mathbf{\tilde{u}}_N )}{\ln{(1+\Delta/z_0)}} \right)^2 & \text{in $\Gamma_{\mathrm{b}} \times [0,T]$} \, .
	\label{eq_motions}
\end{cases}
\end{equation}
%
Here $\tilde{u}_i$ are the filtered velocity components in the three coordinate directions, $\tilde{\pi}$ is a modified filtered pressure field, namely $  \tilde{\pi} = \frac{\tilde{p}}{\rho} + \frac{1}{3}\tau_{ii}^{SGS} + \frac{1}{2}\tilde{u}_i\tilde{u}_i $, $\rho$ is a reference density, $\tau_{ij}^{SGS}$ is the subgrid-scale (SGS) stress tensor (resulting from the filtering operation \citep{Pope2000a}), $ \Pi_1 = \frac{1}{\rho} \frac{\partial \tilde{p}_{\infty}}{\partial x_i}\delta_{i1} < 0$ is a pressure gradient that is introduced to drive the flow, and $\tilde{f}_i^{\Gamma_{\mathrm{b}}}$ is a forcing term that is used to impose the desired boundary condition at the surface location. $\tilde{f}_i^{\Gamma_{\mathrm{b}}}$ has a finite value at the buildings interface $(\Gamma_{\mathrm{b}})$ and is zero elsewhere.
Further, $\mathbf{\tilde{t}}$ is the stress vector at the surface location, $\mathbf{\tilde{u}}_N$ is the normal-to-surface velocity vector, $\Delta = (dx \times dy \times dz)^{1/3}$ and $z_0$ is the hydrodynamic roughness length parameter.
The argument of the $\log$ function in Eq. \ref{eq_motions} has been regularized by adding a unity constant \citep{Chester2007}.

The LES algorithm has been previously used to study land atmosphere interaction processes \citep{Albertson1999a, Albertson1999} and to develop and test linear and non-linear LES subgrid-scale models \citep{Meneveau1996a, Porte-Agel2000, Port2004, Bou-Zeid2005, Lu2010, Lu2013}.

Equations are solved in strong form on a regular domain $\Omega$, a pseudo-spectral collocation approach \citep{Orszag1969a, Orszag1970} based on truncated Fourier expansions is used in the $x,y$ coordinate directions, whereas a second-order accurate centered finite differences scheme is adopted in the vertical direction, requiring a staggered grid approach for the $\tilde{u},\tilde{v},\tilde{p}$ state variables (these are stored at $(j+1/2)dz$, with $j=1,nz$).
Time integration is performed adopting a fully explicit second-order accurate Adams-Bashforth scheme and a fractional step method \citep{Alexandre1968, Fluid1985} is adopted to compute the pressure field, which is based on an operator-splitting technique.
In addition, nonlinear terms are deliased via the $3/2$ rule \citep{Canuto2006}, to avoid piling up of energy in the high wavenumber range \citep{Kravchenko1997a}.
The computational boundary is partitioned as $\partial \Omega = \Gamma_{\mathrm{b}} \cup \Gamma_{\mathrm{top}} \cup \Gamma_{lateral}$, where $\Gamma_{\mathrm{top}}$ and $\Gamma_{lateral}$ denote the top and lateral boundaries respectively. A free-lid boundary condition applies at $\Gamma_{\mathrm{top}}$ and a parameterized boundary condition is prescribed at $\Gamma_{\mathrm{b}}$ (see in Eq. \ref{eq_motions}). 
Periodic boundary conditions apply at $\Gamma_{lateral}$ due to the Fourier spatial representation.

\subsubsection{Subgrid-scale closure model}
The proposed study considers two LES closure models to evaluate $\tau_{ij}^{SGS}$: the classical static Smagorinsky model \citep{Smagorinsky1963} in conjunction with a wall damping function (SMAG), similar to that adopted in \cite{Mason2006}, and the scale-dependent model with Lagrangian averaging of the coefficient (LASD), developed in \cite{Bou-Zeid2005}. 

Smagorinsky models rely on the viscous analogy and on the mixing length concept, and evaluate the subgrid-scale terms as a function of the resolved strain rate tensor: 
%
\begin{linenomath*}
\begin{equation}
  \tau_{ij}^{SGS} = - 2 \nu_t \tilde{S}_{ij} = -2  (c_{s,\Delta} \Delta)^2 \| \tilde{S} \|_2 \tilde{S}_{ij} \, ,
    \label{statSmag}
\end{equation}
\end{linenomath*}
%
where $\nu_t$ represents the eddy viscosity, $\Delta$ is the filter width (usually proportional to the grid size), $\tilde{S}_{ij}$ is the filtered shear rate tensor and $c_{s,\Delta}$ is the Smagorinsky coefficient at scale $\Delta$.
The two models essentially differ in the way they compute the Smagorinsky coefficient.

The SMAG model prescribes a constant coefficient, whose value is usually the one derived from the theory of homogeneous turbulence $(c_{s,\Delta}=0.16$, for the sharp spectral cutoff filter).
However, in applications involving high Reynolds number boundary layer flows -- such as the one proposed herein -- the model is known to be over-dissipative in the near wall regions,  where $c_{s,\Delta}$ should approach zero. To cope with this we introduce an empirical wall damping function \citep{Mason2006}, which has the drawback of requiring an ad-hoc calibration for each specific flow case, but partially ameliorates the dissipative properties of the SMAG model.


The LASD model overcomes the necessity of ad hoc specification of the damping function by exploiting the smallest resolved scales to compute the model coefficient at runtime. It represents an evolution of the original dynamic model, based on the Germano identity \citep{Germano1991a} and its modifications \citep{Lilly1992}. 
LASD relaxes the scale invariance assumption of the model coefficient, which is a desirable property in the near wall regions, where the grid size approaches the limits of the inertial subrange \citep{Meneveau2000}.
The Lagrangian averaging of the model coefficient makes the model well suited for applications involving complex geometries, since it preserves local variability while satisfying Galileian invariance, and overcomes the requirement of homogeneous directions \citep{Bou-Zeid2005}. 
Additionally, along fluid pathlines the energy cascade process is more apparent \citep{Meneveau1994}, which enforces the theoretical basis of the model.
To reduce the strong Gibbs oscillations that would arise at the interface if adopting a classic spectral cutoff filter, a Gaussian filter is introduced in conjunction with the LASD model, which has the desirable property of being of compact support in both physical and wavenumber space \citep{Tseng2006}.


\subsubsection{Discrete forcing immersed boundary method (IBM)}

To model the urban canopy a discrete forcing approach immersed boundary method is adopted \citep{Mohd-yusof1997,Mittal2005}. 
The buildings' interface $\Gamma_{b}(x,y)$ is represented implicitly as the zero level-set of a (higher dimensional) signed distance function $\tilde{\phi}(x,y,z)$, and the computational domain $\Omega$ is splitted in two regions: the inside building region $\Omega_b$, where $\tilde{\phi} \leq 0$, and the fluid region $\Omega_f$, where $\tilde{\phi} > 0$.
The $\tilde{\phi}(x,y,z)$ function is initialized adopting an iterative projection technique on the  triangulated (urban) surface, which has been specifically developed for the current study.
The immersed boundary algorithm is a minor modification of the one proposed in \cite{Chester2007}. 
The velocity field is fixed to zero in $\Omega_b$ through a penalty method and the law of the wall is enforced at all the collocation nodes which fall in the region $-1.1 \Delta \le \tilde{\phi} \le 1.1 \Delta$. 
The law of the wall is based on the equilibrium logarithmic assumption \citep{Moeng1984} and computes the local surface stress vector as 
%
\begin{linenomath*}
\begin{equation}
      \mathbf{\tilde{t}} = - \left( \frac{ \kappa ( \mathbf{\tilde{u}} - \mathbf{\tilde{u}}_N )}{\ln{(1+\Delta/z_0)}} \right)^2 \, .
\end{equation}
\end{linenomath*}
%
The main difficulty in coupling the immersed boundary method with a pseudo-spectral algorithm is represented by the fact that the domain is not simply connected. 
The solution to Eqs. \ref{eq_motions}, in a given plane cutting the building elements, is of class $C^0$, with the discontinuities in first derivatives localized at the building-atmosphere interface $\Gamma_{b}$. The spectral representation will result in Gibbs oscillations in the near interface regions, which will then propagate away from the singularity and degrade the quality of the partial sum approximation \citep{Geer1997}.
To alleviate such phenomena a smooth velocity profile $\tilde{u}_i$ is generated in $\Omega_b$ ($\tilde{\phi} \le 0$) before the spectral differentiation step \citep{Tseng2006} adopting a Laplacian smoothing operator which resembles the reconstruction scheme proposed in \citet{Cai1989} and \citet{Geer1997}. 
Alternative smoothing algorithms are also available, as in \citet{Fang2011} and \citet{Li2016}.

%%%%%%%%%%%%%%%%%%%%%%%%%%%%%%%%%%%%%%%%%%%%%%%%%%%%%%%%%
\subsection{Site description and instrumentation}

Numerical solutions are compared to field data from the Basel Urban Boundary Layer Experiment (BUBBLE), a multi-institutional effort dedicated to the energetics and dispersion processes in the urban boundary layer \citep{Rotach2005}.
%
\begin{figure}[h]
	\centering
%	\begin{tabular}{c c}
      		\includegraphics[width= 120.0mm]{./reference_surface_contour} 
%        \end{tabular}
     	\caption{\label{basel1} Color contour of the surface height $(\Gamma_{\mathrm{b}}(x,y))$ for a neighborhood scale of $512\times512 \ \mathrm{m}$, centered at the tower location. The Sperrstrasse street canyon is aligned with the $x$ coordinate axis.}
\end{figure}

%		\includegraphics[width= 25.0mm]{images/tower_picture}

%
During BUBBLE, a $32 \ \mathrm{m}$ high tower was deployed inside the $13 \ \mathrm{m}$ wide Sperrstrasse street canyon in Basel, Switzerland $(47^{\circ}33'57.20"\mathrm{N}, 7^{\circ}35'48.80"\mathrm{E}, \mathrm{WGS}-84)$, as displayed in Fig. \ref{basel1}. 
The orientation of the street canyon is along the axis $066^{\circ} - 246^{\circ}$ (ENE to WSW), the block where the tower was operated is characterized by a length of $160 \ \mathrm{m}$, and an average width-to-height ratio of $\xi_c / z_h = 1.0$, where $\xi_c$ is the street canyon width and where $z_h$ is the mean building height. 
%
\ctable[cap = {Details on the tower instrumentation},
           caption = {Details on the ultrasonic anemometer-thermometer (sonic) instrumentation, label, absolute measurement heights $z$, normalized measurement heights (the normalization scale is the location of the highest sonic), sonic type, sampling frequency $f \ (\mathrm{Hz})$.},
            label={sonicTable},
            doinside=\small \renewcommand\arraystretch{1.2},
            pos = ht,
            width   = 110mm,]{>{\hsize=.5\hsize}X >{\hsize=.5\hsize}X >{\hsize=.5\hsize}X >{\hsize=1.2\hsize}X >{\hsize=.5\hsize}X}
{
  % You specify table footnotes here.
}{
\FL % FLORIDA (just kidding, means "first line")s
    Label & $z \ (\mathrm{m})$ & $z/z_t$ & Instrument type & $f \ (\mathrm{hz})$ 		\ML % middle line
    %
    $A$ & 3.6 & 0.11 & Gill R2 Omnidirectional & 20.8  			\\
    %
    $B$ & 11.3 & 0.35 & Gill R2 Omnidirectional & 20.8 			\\
    %
    $C$ & 14.7 & 0.46 & Gill R2 Omnidirectional & 20.8			\\
    %
    $D$ & 17.9 & 0.56 & Gill R2 Omnidirectional & 20.8			\\
    %
    $E$ & 22.4 & 0.7 & Gill R2 Asymmetric & 20.8				\\
    % 
    $F$ & 31.7 & 1 & Gill HS & 20.0                              			\LL 
}
%
The tower was placed at the midpoint of the block, $3 \ \mathrm{m}$ away from the north wall, and equipped with six ultrasonic anemometer-thermometers (sonics, labels $A-F$ in table \ref{sonicTable}), mounted on horizontal booms reaching from the tower into the centre of the street canyon. 

%
% SITE DESCRIPTION
%
Buildings on both sides of the street canyon ``Sperrstrasse" have pitched roofs except two
flat-roof buildings directly adjacent to the tower on the northern side (label 1 and 2 in Fig. \ref{basel1}) and two flat-roof buildings close to the two intersections (label 3 and 4). The height of buildings typically reaches $15 \ \mathrm{m}$ on both sides. A high pitched roof of $20 \ \mathrm{m}$ is located directly to the south-east of the tower (label 5) \citep{Christen2009}.
Sectors from west to NNE and SSE to SSW are similar to structures found immediately around the tower.
These sectors are homogeneous in terms of integral morphometric statistics and building height with fetch extending to $700 \ \mathrm{m}$. In the sector NE to SSE an extensive commercial area is found at $100 \ \mathrm{m}$ distance to the tower with flat roofs and roof heights from $20 \ \mathrm{m}$ to $25 \ \mathrm{m}$ (label 6), whereas an isolated high-rise building of $64 \ \mathrm{m}$ height is located $\approx 200 \ \mathrm{m}$ to the south-west of the tower (label 7). 
A $18.5 \ \mathrm{m}$ building is located approximately $100 \ \mathrm{m}$ north-east of the tower (label 8).
For the considered neighborhood, trees are all of the same height and lower than buildings, and the plan area fraction of vegetation (grass plus trees) is only 0.16 \citep{Christen2004}.


\subsection{The urban canopy dataset}
A high resolution three dimensional terrain and building digital model (vector format) which includes downtown and sub-urban areas of Basel, was provided by the authorities of the city (GVA Grundbuch und Vermessungsamt Basel-Stadt). 
The building model includes details such as openings and chimneys, but does not include vegetation. Neglecting vegetation is justified considering its small plan area fraction (0.16).
The dataset was rasterized at a horizontal resolution of $0.5 \ \mathrm{m}$ and rotated by $-24^{\circ} (\text{clockwise})$ in order to have the main street canyon aligned with the coordinate system $(x,y,z)$, as in Fig. \ref{basel1}. 
The pdf of roof heights is characterized by a trimodal distribution (see left plot in Fig. \ref{roofs_stats}) with modes at $z \approx 4.5 \ \mathrm{m}$ (Mo$_1$), $z \approx 17.5 \ \mathrm{m}$ (Mo$_2$) and $z \approx 22.5 \ \mathrm{m}$ (Mo$_3$).
The mean roof height $z_h$ is $15.3 \ \mathrm{m}$ and the variance of roof height is $6.4 \ \mathrm{m}$. 
The first mode Mo$_1$ corresponds to one-storey buildings in the backyards (garages, commercial buildings, etc.), the second mode Mo$_2$ is related to the the main residential (attached) buildings that line streets and enclose courtyards, whereas the third mode Mo$_3$ is linked to building N.6 in Fig. \ref{basel1}, whose large surface has significant impact on the pdf of the surface heights.
%
\begin{figure}
	\centering
        \begin{tabular}{c c}
        \includegraphics[width= 75.0mm]{./histo_roofs} &
        \includegraphics[width= 30.0mm]{./plan_area_fraction} 
       \end{tabular}
       \caption{Binned pdf of the roof heights (left) and plan area fraction $\lambda_p(z)$ (right) for the considered surface ($512 \times 512 \ \mathrm{m}$). }
      \label{roofs_stats}
\end{figure}
%


\subsection{Processing of the profile tower dataset}

Wind components $u,v,w$ and virtual acoustic temperature $\theta$ were continuously recorded at all six levels simultaneously from December 1, 2001 to July 15, 2002.
Data acquisition systems and quality control procedures including wind-tunnel calibrations of the instruments are described and documented in \cite{Christen2005}.
$u,v$ and $w$ statistical moments up to order three were calculated and stored for blocks over 5 minutes. No filtering was applied to the signal nor standard de-trending, to ensure energy conservation and enable vertical gradients of the state variables to be properly computed. 
To provide data for comparison with pressure driven simulations the following processing is further performed:
%
\begin{enumerate}
	\item data are averaged in blocks of 30 minutes;
	\item data are selectively sampled from the year-round dataset based on the wind direction computed at the tower top sensor. Only $30$ min blocks characterized by an approaching wind direction of $\alpha =  66^{\circ} \pm 10^{\circ}$ (along-canyon regime) and of $\alpha = 156^{\circ} \pm 10^{\circ}$ (across-canyon regime) throughout the $6 \times 5$ min intervals are kept;
	\item in order to eliminate the influence of thermal stability, the periods are further filtered based on the classic stability parameter $\zeta = (z - z_d)/L$ \citep{Stull1988}, where $L$ is the Obukhov length $(L = \theta u^2_{\tau} / [\kappa g \theta_*])$ calculated with both friction velocity $u_{\tau}$ and scaling temperature $\theta_*$ measured at the tower top. Only periods characterized by near-neutral stability are kept, $-0.1 \le \zeta \le +0.1$. The displacement height is computed as $z_d=2/3 z_h$, in the typical range suggested for high-density urban roughness elements \citep{Grimmond1999};
	\item cases characterized by $u_{\tau} \le 0.15 \ \mathrm{ms^{-1}}$ at tower top are excluded from the analysis.
\end{enumerate}
%
Despite the strict constraints, the availability of a relatively long dataset resulted in $30$ blocks for the ENE wind approaching direction ($\alpha =  66^{\circ} \pm 10^{\circ}$) and $3$ blocks for the SSE wind approaching direction ($\alpha =  156^{\circ} \pm 10^{\circ}$).


\subsection{Setup of simulations}

%
\ctable[cap = {Geometrical and numerical parameters for the LES runs. },
           caption = {Geometrical and numerical parameters for the LES runs.},
            label={domainSizeParam},
            doinside=\small \renewcommand\arraystretch{1.0},
            pos = ht,
            width   = 100mm,]{>{\hsize=.25\hsize}X >{\hsize=.25\hsize}X >{\hsize=.25 \hsize}X >{\hsize=.25 \hsize}X}
{
  % You specify table footnotes here.
  \tnote[]{}
}{
\FL % FLORIDA (just kidding, means "first line")
     ID & $ z_0 \ \mathrm{(m)}$ & $\alpha$ (deg) & SGS model \ML   %
    $A$ &  $\Delta/15$ & $66$ & SMAG \\
    $B$ & $\Delta/30$ & $66$  & SMAG   \\
    $C$ &  $\Delta/15$ & $156$  & SMAG \\
    $D$ & $\Delta/30$ & $156$ & SMAG   \\
    $E$ &  $\Delta/15$  & $66$& LASD \\
    $F$ &  $\Delta/30$ & $66$  & LASD   \\
    $G$ &   $\Delta/15$ & $156$  & LASD \\
    $H$ &  $\Delta/30$ & $156$ & LASD   \LL
    }
	Simulations are performed over a regular domain, of size $L_x \times L_y \times L_z = 512 \times 512 \times 160$, (horizontally) centered at the tower locations $(x_t,y_t)$ and discretized with $1 \, \mathrm{m}$ stencil in the three coordinate directions ($x,y,z$). 	
	Numerical parameters for each run are summarized in Table \ref{domainSizeParam}. 
	Two directions of the incoming wind are considered, $\alpha = 66^{\circ}$ and $\alpha = 156^{\circ}$, which correspond to an along-canyon and across-canyon wind regime respectively. 
	The flow is forced by imposing a constant pressure gradient $\partial_x p_{\infty} /\rho$, which, in conjunction to lateral periodic boundary conditions, defines a friction velocity $u_{\tau} = \sqrt{(\delta-z_d)\partial_x p_{\infty}/\rho}  \approx 1.23 \ \mathrm{m \ s\textsuperscript{-1}}$, making the system independent from Reynolds number effects (fully rough flow regime). 
	Under such conditions it is possible to scale the solution throughout the boundary layer with a characteristic velocity $U$, since molecular diffusion is negligible.
	The relatively homogeneous integral morphometric statistics and buildings height in the neighborhood justifies the pressure forcing in conjunction with lateral periodic boundary conditions  (the main surface transition occurs at $\approx 700 \ \mathrm{m}$ in the radial direction from the tower location).	Domain size was chosen based on a sensitivity study (not shown).
%, adopting Wood's equation for the growth of the internal boundary layer (IBL) \cite{Wood1982a}
%%
%\begin{equation}
%     z_{H}^{IBL}/k_{or} = 0.28 \left(  x_i/k_{0r} \right)^{0.8},
%\end{equation}
%%
%where $z_{H}^{IBL}$ is the height of the IBL, $x_i$ is the stream-wise coordinate direction and $k_{0r}$ is the larger of the surface roughnesses $k_{01}, k_{02}$ characterizing the transition, would result in $\partial z_{H}^{IBL}/ \partial x_i \approx 0.05m/m$ and $\langle z_{H}^{IBL} \rangle=50m$. The growth of the IBL is therefore a minor percentage of the average $z_{H}$ in the domain of interest and justifies the approach.
	The hydrodynamic roughness length $z_0$, defining the surface roughness, is not known a priori. Here, $z_0$ is defined based on a Nyquist-Shannon representation criteria \citep{Shannon1949}: adopting a reference grid stencil $\Delta$, the smallest flow/surface feature that is representable through the Fourier partial sums is $k_{\Delta} = 2\Delta$, whereas all scales smaller than $k_{\Delta}$ need to be modeled. 
	Since $z_0 = 0.033 k_s$, where $k_s$ is the equivalent Nikuradse sand grain roughness, and given that $k_s \rightarrow k$ in the limit of negligible viscous effects ($k$ is the height of the considered roughness element), we have that $z_0 = 0.033 k_{\Delta} \approx \Delta/15$. 
	To account for variations in the solution due to the $z_0$ parameter $z_0= \Delta / 30$ is also considered.
	To reduce the computational time required to reach a dynamic equilibrium, the initial velocity field for each simulation is imposed through interpolation from results of a run at coarser resolution (twice as coarse in each coordinate direction). 
	Equations are integrated in time for $480$ non-dimensional time units $T = z_h/u_{\tau}$ ( $\approx 2 \ \mathrm{hours}$ in dimensional time) in the coarser grid, before being used as initial condition for the finer grid, where they are further integrated for $250T$. 
$100T$ are required in order to achieve statistical stationarity in the velocity field and $150T$ are used to compute statistics, which ensures convergence of first and second order moments to the corresponding expected values.
	To further reduce computational costs the $\delta/z_h \gtrapprox 50$ requirement \citep{Jimenez2004} is here sacrificed. Simulations are characterized by $\delta/z_h = 10.6$. Roughness has a great influence on turbulence up to ${z/z_h \approx \min(1+D/z_h,5)}$, where $D$ is the separation distance between nearest-neighbour roughness elements \citep{Raupach1981, Jimenez2004}. Assuming the top of the RSL to be located at $z/z_h=5$ implies that the current geometry does not allow an ISL to survive. 
	The limited $\delta /z_h$ in the proposed study might be justified by considering that the focus is on the dynamics within the RSL.
In these regions turbulence is expected to be strongly affected by the morphology of the roughness elements and only in minor part by dynamics of the logarithmic and outer layer \citep{Anderson2016a}.